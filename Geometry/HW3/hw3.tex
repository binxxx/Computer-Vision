% Scribe template is a combination of 16831 and 10725. Thanks to those TAs!
\documentclass[11pt]{article}
\usepackage{latexsym}
\usepackage{amsmath}
\usepackage{amssymb}
\usepackage{amsthm}
\usepackage{bm}
\usepackage{epsfig}
\usepackage[tight]{subfigure}
\usepackage{hyperref}
\usepackage{listings}


\newcommand{\handout}[5]{
  \noindent
  \begin{center}
  \framebox{
    \vbox{
      \hbox to 5.78in { {#1} \hfill #2 }
      \vspace{4mm}
      \hbox to 5.78in { {\Large \hfill #5  \hfill} }
      \vspace{2mm}
      \hbox to 5.78in { {\em #3 \hfill #4} }
    }
  }
  \end{center}
  \vspace*{4mm}
}

\newcommand{\lecture}[5]{\handout{#1}{#2}{#3}{#4}{#5}}

\newtheorem{theorem}{Theorem}
\newtheorem{corollary}[theorem]{Corollary}
\newtheorem{lemma}[theorem]{Lemma}
\newtheorem{observation}[theorem]{Observation}
\newtheorem{proposition}[theorem]{Proposition}
\newtheorem{definition}[theorem]{Definition}
\newtheorem{claim}[theorem]{Claim}
\newtheorem{fact}[theorem]{Fact}
\newtheorem{assumption}[theorem]{Assumption}

% 1-inch margins, from fullpage.sty by H.Partl, Version 2, Dec. 15, 1988.
\topmargin 0pt
\advance \topmargin by -\headheight
\advance \topmargin by -\headsep
\textheight 8.9in
\oddsidemargin 0pt
\evensidemargin \oddsidemargin
\marginparwidth 0.5in
\textwidth 6.5in

\parindent 0in
%\parskip 1.5ex
\renewcommand{\baselinestretch}{1.1}

\begin{document}
\newcommand{\defeq}[0]{\ensuremath{\stackrel{\triangle}{=}}}
\def\x{\mathbf{x}}
\def\w{\mathbf{w}}
\def\K{\mathbf{K}}
\lecture{{\bf 16-822}: Geometry-based Methods in Vision (F17) }{Released: Oct-25., Due: Nov 15th}{Lecturer: Martial Hebert}{TA: Aayush Bansal}{Homework 3}


\section{Affine cameras (35 Points)}
In class, we have mostly focused on perspective cameras. For this part of the homework, we consider cameras with center lying on the plane at infinity, especially {\it affine cameras}. An affine camera is one that has a camera matrix of the form $M=\left[A \,|\, b\right]$ in which A is singular and the last row of M is of the form $(0, 0, 0, 1)$. The following questions explore affine cameras a little more in depth.
\begin{itemize}
\item What happens to points on a plane perpendicular to the image plane that also passes through the world origin?
\item Given a set of points $X_i$ in $3D$, and their projection $x_i$ in the image, what is the projection of the centroid of $X_i$'s? (no equations, just 1 short line.)
\item Show that an affine camera maps parallel world lines to parallel image lines; i.e., when using an affine camera, parallel lines in $3D$ project to parallel lines in image plane.\footnote{Actually, the converse is also true; if the camera preserves parallelism of lines, then it is an affine cameras.}
\item Show that for parallel lines mapped by an affine camera, the ratio of lengths on line segments is an invariant.
\end{itemize}
For the following questions, consider a system of $2$ affine cameras, with camera matrix $M$ and $M'$: (Last row of each camera matrix is of the form $(0, 0, 0, 1)$.)
\begin{itemize}
\item Show that epipolar planes (and lines) are parallel.
\item Show that the affine fundamental matrix $F$ defined by two affine cameras is invariant to an affine transformation of the world coordinates.
\end{itemize}


\section{Two camera geometry (10 + 20 + 10 Points)}
\subsection{2 singular values of $E$ are equal (10 Points)}

Let $E$ be the {\em essential} matrix of a pair of cameras. We know that $E$ is singular so that one of its singular value is zero. 
Show that the other 2 singular values are equal. (Remember that the singular values of a matrix $A$ are the square roots of the {\em eigenvalues}
of the matrix $AA^T$. Apply this property to $E$ by apply $EE^T$ to the appropriate vectors to figure out what the eigenvalues are.)

\subsection{Recovering $R$, $t$ from $E$ (20 Points)}
We address the problem of recovering the rotation $R$ and translation $t$ from
the essential matrix $E = [t]_\times R$.
\begin{enumerate}
\item Show how to recover $t$ from $E$ (one single equation).
\item Show (using the properties of the essential matrix stated in class), that
the SVD of $E$ is of the form: $E = U\texttt{diag}(1,1,0)V^T$.
\item Show that the two matrices $R_1 = UWV^T$ and $R_2 = UW^TV^T$ are
solutions in $R$, where $W = [0\ {-1}\ 0;\ 1\ 0\ 0;\ 0\ 0\ 1]$.
\item Why is there 2 solutions in $R$ (in words or drawing)?
\end{enumerate}

\subsection{Extra Credits - $\Pi$ $\|$ to the baseline (10 Points) }
Consider a plane $\Pi$ viewed in a two-camera system. Show that if the vanishing line of $\Pi$ in one image (i.e., the projection of the line at infinity of $\Pi$ in the image) contains the epipole in that image, then $\Pi$ is parallel to the baseline between the 2 cameras (the line joining the 2 camera centers).



\section{Fundamental matrices between three cameras (25 Points)}
In this problem we consider the relationship between $3$ images. Let us denote by $F_{ij}$ the fundamental matrix between image $i$ and $j$ (from 1 to 3), and by $e_{ij}$ the epipole generated by image $j$ in image $i$. We will now show that the $3$ fundamental matrices induced by $3$ cameras are not independent. 

\begin{itemize}
\item Draw the 3 cameras with $e_{ij}$ labelled.
\item Show that: $F_{ij}e_{ik} = e_{jk} \times e_{ji}$ for all triplets. (this can be shown purely geometrically, without 
involved algebra)
\item Based on these 3 quadratic relations, give a counting argument showing that the 3-camera geometry is characterized by 18 degrees of freedom. And thus conclude that the $3$ fundamental matrices are not independent.
\end{itemize}

\section{Auto-calibration (20 + 10 + 10)}
\subsection{General Questions (20 Points)}
\begin{enumerate}
\item No equations, use a counting argument only: Suppose that we have
$m$ images from which we have generated a projective reconstruction.
Suppose that the intrinsic parameters ($K_i$) are fixed throughout all of
the images but unknown except for the skew which is known to be $0$.
How many images are needed to solve for the calibration parameters
(auto-calibration toward metric reconstruction)?
\item Can it be solved in a linear fashion in this case?
\item Suppose that all the calibration parameters in $K$ are allowed to vary
across the images, but that we know that the two coordinates of the
principal point are equal in each of the images: $x_o = y_o$. How many
images are needed to solve the auto-calibration problem?\footnote{FYI, this particular constraint does not correspond to a realistic physical constraint on the cameras, but it is still a valid constraint in theory.}
\item Can it be solved linearly?
\item Linear approaches to the auto-calibration problem solve for $L = Q_3Q^T_3$,
where $Q_3$ is the $4 \times 3$ matrix of the first three columns of the $4 \times 4$
rectifying matrix $Q$. After solving for $L$, how can we recover $Q_3$?
\end{enumerate}

\subsection{Extra Credits - Auto-calibration, special case (10 Points)}
We showed that, given m cameras, if we know the homography of the plane
at infinity, $H^i_\infty$, between the first image and each image $i$, we can derive
the relations: $\omega^* = H^i_\infty \omega^* (H^i_\infty)^T$, where $\omega^* = KK^T $, assuming constant
intrinsic parameters. Although, in principle we can solve for $K$ given enough
equations, there are cases in which the problem degenerates and an infinite
family of solutions exist.
\begin{enumerate}
\item Does the homography of the plane at infinity between any two images
depend on the translation between the images?
\item Show that, if the cameras are in translation only without rotation, the
auto-calibration problem cannot be solved using the relations based
on the plane at infinity as shown above (basically no equations, use
the result of the previous question)\footnote{FYI, in fact, not only we need some amount of rotation to be able to solve unambiguously, we need rotations about at least three different axes. In other words, the camera
needs to ``move enough'' to be able to self-calibrate unambiguously!}.
\end{enumerate}

\subsection{Extra Credits - Weird constraints for auto-calibration (10 Points)}
\begin{enumerate}
\item How many views $m$ are needed to generate a metric reconstruction assuming that 1) the skew is zero and 2) the
principal point is constrained to be on a circle of radius $r$ centered at the origin $(0,0)$?
\item Can it be solved linearly?
\end{enumerate}



\end{document}
